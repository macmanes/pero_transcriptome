% Template for PLoS
% Version 1.0 January 2009
%
% To compile to pdf, run:
% latex plos.template
% bibtex plos.template
% latex plos.template
% latex plos.template
% dvipdf plos.template

\documentclass[10pt]{article}

% amsmath package, useful for mathematical formulas
\usepackage{amsmath}
% amssymb package, useful for mathematical symbols
\usepackage{amssymb}

% graphicx package, useful for including eps and pdf graphics
% include graphics with the command \includegraphics
\usepackage{graphicx}

% cite package, to clean up citations in the main text. Do not remove.
\usepackage{cite}

\usepackage{color} 

% Use doublespacing - comment out for single spacing
%\usepackage{setspace} 
%\doublespacing


% Text layout
\topmargin 0.0cm
\oddsidemargin 0.5cm
\evensidemargin 0.5cm
\textwidth 16cm 
\textheight 21cm

% Bold the 'Figure #' in the caption and separate it with a period
% Captions will be left justified
\usepackage[labelfont=bf,labelsep=period,justification=raggedright]{caption}

% Use the PLoS provided bibtex style
\bibliographystyle{plos2009}

% Remove brackets from numbering in List of References
\makeatletter
\renewcommand{\@biblabel}[1]{\quad#1.}
\makeatother


% Leave date blank
\date{}

\pagestyle{myheadings}
%% ** EDIT HERE **


%% ** EDIT HERE **
%% PLEASE INCLUDE ALL MACROS BELOW

%% END MACROS SECTION

\begin{document}

% Title must be 150 characters or less
\begin{flushleft}
{\Large
\textbf{Title}
}
% Insert Author names, affiliations and corresponding author email.
\\
Matthew D. MacManes$^{1}$, 
Author2$^{2}$, 
Author3$^{3,\ast}$
\\
\bf{1} Author1 Dept/Program/Center, Institution Name, City, State, Country
\\
\bf{2} Author2 Dept/Program/Center, Institution Name, City, State, Country
\\
\bf{3} Author3 Dept/Program/Center, Institution Name, City, State, Country
\\
$\ast$ E-mail: Corresponding author@institute.edu
\end{flushleft}

% Please keep the abstract between 250 and 300 words
\section*{Abstract}



\section*{Introduction}

For biologists interested in understanding the relationship between fitness, genotype, and phenotype, modern sequencing technologies provide for an unprecedented opportunity to gain a deep understanding of genome level processes that together, underlie adaptation.  One interesting example of adaptation lies in animals ability to survive desert conditions. Here, heat \emph{and} drought provide for powerful selective forces, testing animals' ability to osmoregulate and thus to survive. \\

Specifically, the maintenance of water balance in animals is one of the most important physiologic processes, and is critical to desert survival. Indeed, mammals are exquisitely sensitive to changes in osmolality, with slight derangement eliciting physiologic compromise.  When the loss of water exceeds dietary intake, dehydration - and in extreme cases, death - can occur.  Unlike most mammals, animals living in desert habitats are subjected to long periods of extreme heat and intense drought.  As a result, desert animals have evolved mechanisms through which physiologic homeostasis is maintained despite severe and prolonged dehydration. \textbf{The proposed research uses a novel approach integrating physiology, evolutionary genomics, and computational biology to better understand how animals survive in what appear to be unsurvivable conditions.} This integrative research program will significantly enhance our understanding of the physiologic processes underlying osmoregulation in extreme environments. Its impact is broad, having implications for studies of human health, conservation, and climate change. This project both leverages and enhances existing ecological, behavioral, and genomic resources available to an active \textit{Peromyscus} research community, ultimately allowing for interesting studies of comparative functional biology not possible in traditional model organisms like \textit{Mus}. 


% Results and Discussion can be combined.
\section*{Results}

\subsection*{Subsection 1}

\subsection*{Subsection 2}

\section*{Discussion}

% You may title this section "Methods" or "Models". 
% "Models" is not a valid title for PLoS ONE authors. However, PLoS ONE
% authors may use "Analysis" 
\section*{Materials and Methods}

% Do NOT remove this, even if you are not including acknowledgments
\section*{Acknowledgments}


%\section*{References}
% The bibtex filename
\bibliography{template}

\section*{Figure Legends}
%\begin{figure}[!ht]
%\begin{center}
%%\includegraphics[width=4in]{figure_name.2.eps}
%\end{center}
%\caption{
%{\bf Bold the first sentence.}  Rest of figure 2  caption.  Caption 
%should be left justified, as specified by the options to the caption 
%package.
%}
%\label{Figure_label}
%\end{figure}


\section*{Tables}
%\begin{table}[!ht]
%\caption{
%\bf{Table title}}
%\begin{tabular}{|c|c|c|}
%table information
%\end{tabular}
%\begin{flushleft}Table caption
%\end{flushleft}
%\label{tab:label}
% \end{table}

\end{document}

